\documentclass{article}
\usepackage[utf8]{inputenc}
\usepackage{geometry}
\usepackage{natbib}
\usepackage{pxfonts}

\title{Episodic memory: mental time travel or a quantum wave function?}
\author{Jeremy R. Manning\\Dartmouth College, Hanover, NH\\jeremy.r.manning@dartmouth.edu}

\begin{document}
\maketitle

\begin{abstract}
Where do we ``go'' when we recollect our past?  When remembering a past event, it is intuitive to imagine some part of ourselves mentally ``jumping back in time'' to when the event occurred. I propose an alternative view, inspired by recent evidence from my lab and others, as well as by re-examining existing models of episodic recall, that suggests that this notion of mentally revisiting any specific moment of our past is at best incomplete and at worst misleading.  Instead, I suggest that we retrieve information from our past by mentally casting ourselves back simultaneously to \textit{many} time points from our past, much like a quantum wave function spreading its probability mass over many possible states.  This revised conceptual model makes important behavioral and neural predictions about how we retrieve information about our past, and has implications for how we study episodic memory experimentally.
\end{abstract}

\section*{Introduction and overview}
How do our brains organize, retrieve, and act upon the ceaseless flow of incoming information and experiences?  \cite{Tulv83} contends that episodic memory is like a sort of ``mental time travel,'' whereby we send some part of our mental state back along our autobiographical time line to a specific moment in our past.  This view has fundamental implications for how we study and build theories about memory: it suggests that we can match up specific moments of recollection with specific moments from the past.  These implications extend to how we design episodic memory experiments and how we analyze the behavioral and neural data from those experiments in order to understand the neural and/or cognitive mechanisms that underly memory retrieval.  At its core, characterizing episodic memory as mental time travel makes studying and understanding memory about matching up the thoughts (and/or brain activity patterns) participants are experiencing at one moment with the thoughts (and/or brain activity patterns) participants experienced at another previous moment.  The challenge, then, is to understand the neural (structural) and cognitive (functional) mechanisms that characterize when, how, and why our memory systems allow us to relive those prior experiences.  One way of illustrating how this framing misses out on key aspects of memory is to examine how experimentalists and theorists typically conceive of and model a mental construct called \textit{context}.

The notion of context was fundamental to \citeauthor{Tulv83}'s characterization of episodic recall, and continues to be a primary feature of most current theories of episodic memory~\citep[e.g.\ for review see][]{Kaha12}.  Whereas the \textit{content} of a memory provides specific information about the episode itself, contextual components of a memory help to frame the episode within the rememberer's broader experience.  Defining context precisely (e.g.\ as distinct from content) is an ongoing challenge for our field, but theorists generally describe context as reflecting a mental combination of external cues (where you are, who you are with, background sensory information, etc.) and internal cues (thoughts, feelings, emotions, etc.) that uniquely define each moment~\citep[e.g.\ see review by][]{MannEtal15}.  Reactivating the mental representation of the context associated with a given episode is what can give us the feeling of some ``part of ourselves'' re-experiencing the past.

One practical way to distinguish content from context representations is to separate mental (or neural) representations that drift rapidly \citep[content;][]{PolyEtal05, MannEtal12} or gradually \citep[context;][]{PolyEtal05, MannEtal11, HowaEtal12, FolkEtal18}.  One intuition for why contextual representation might be expected to evolve gradually is laid out by \cite{PolyKaha08}.  Essentially, slowly drifting thoughts may be used to ``index'' more rapidly drifting thoughts that are temporally proximal.  This notion, that our brain maintains parallel mental representations drifting at different time scales, has been well-characterized in a series of elegant fMRI and ECoG studies by Uri Hasson and colleagues~\citep{HassEtal08, LernEtal11, HoneEtal12}.  Subsequent work by the same group has shown that this hierarchy of drifting mental representations plays a central role in how we remember continuous experiences and segment our experiences into discrete events~\citep{BaldEtal17}. This work mirrors the discovery of \textit{time cells} in the rodent hippocampus that respond preferentially to a specific time relative to a temporal reference, whereby each cell appears to integrate incoming information at a different rate~\citep{PastEtal08, MacDEtal11}. These ideas have also been formalized in a family of theoretical models by Marc Howard and colleagues~\citep{HowaKaha02, SedeEtal08, PolyEtal09, ShanEtal09, ShanHowa10, ShanHowa12, HowaEtal14}.  In turn, these models were inspired in part by earlier theoretical work suggesting that temporally proximal experiences become linked through their shared contextual features~\citep{Este55a,AtkiShif68}.  Collectively, these experimental and theoretical studies make two important contributions to our understanding of how we remember our past.  First, the studies suggest that our ongoing experiences are ``blurred out'' in time by neural integrators (where different brain structures or sub-structures integrate with different time constants).  Second, the studies suggest that members of this family of representations drifting at different rates become bound together such that when we retrieve memories about our past, we reactivate representations that carry information at a range of time scales.  These reactivations explain how our brains reactivate prior contexts when we remember the past.

Although reactivating context is often \textit{described} as a sort of mental time travel back to the one specific moment being remembered, there is some apparent inconsistency between this description and how this process is actually characterized mathematically or measured neurally.  In particular, both the theory and neural measures described in the preceding paragraphs characterize episodic recall mathematically as reflecting the reactivation of a weighted blend of thoughts from a \textit{range} of previously experienced times-- not to any one \textit{specific} time.  This is not simply a matter of imprecision in mental time travel (i.e.\ a reflection of the low effective resolution with which we can revisit specific moments from our past).  Rather, when we retrieve the thoughts associated with a range of times, it is conceptually more like we are simultaneously visiting \textit{many} of our prior experiences. The above human and rodent studies suggest that the brain maintains parallel representations of ongoing experience, each drifting at a different timescale.  The notion of thinking of any one particular moment from the past is incompatible with this multiple timescales view, in that our thoughts and brain activity patterns reflect information from a range of moments, even when we are not specifically engaged in remembering.  If what we consider to be a ``moment'' or ``now'' is guided by our neural representations of our ongoing experiences, then this implies we are continually experiencing \textit{many} ``nows.''  If our thoughts are continually spread across a range of times, the notion of mentally revisiting any particular time loses its meaning.

Where the mental time travel description breaks down most notably is when we consider scenarios in which the particular blend of timepoints reflected in one's current thoughts do not come from temporally contiguous events.  For example, understanding how a new experience fits in with our broader understanding might require integrating information gleaned from many experiences separated in time, each with their own contextual attributes.  That we can integrate information across experiences suggests that the integration of incoming information at different rates need not be a purely mechanical process, but rather may reflect a deeper functional role.  When we simultaneously reactivate thoughts related to separated moments from our past, this provides a means of associating and relating those temporally distinct experiences.  This is how we can integrate information across the \textit{many} discrete experiences relevant to ``now.''

\section*{The temporal dynamics of ongoing experience and how we remember it}
Modern context-based theories of episodic memory posit that the current state of mental context serves to determine which information from our past may be relevant to us and therefore reactivated~\citep[e.g.\ ][]{PolyEtal09}. Although these theories were largely developed and tested in the domain of list-learning paradigms, generally similar principles might be expected to play out in real world memory as well.  For example, prior experiences with shared or overlapping contextual properties (e.g.\ other experiences that occurred in similar spatial or social settings, shared similar goals, etc.) could be leveraged to form situation models that help guide behaviors and expectations according to the current perceived context~\citep{RangRitc12}.

Unlike purely random word lists, naturalistic experiences contain \textit{event boundaries} whereby contextual cues change sharply from one moment to the next, implying a change in the current situation (e.g.\ shifting from the quiet peace of writing a paper to the different peace of consoling a distraught child).  A number of studies (using non-random lists and more naturalistic stimuli) have found that the way we remember past experiences can be shaped by these event boundaries.  For example, controlling for elapsed time, memory is impaired for information that occurred prior to a change in the current event or situation~\citep[e.g.\ ][]{RadvCope06, SwalEtal09, SwalEtal11, EzzyDava11, MannEtal16}.  The temporal dynamics of contextual change also underlie how we judge the amount of time elapsed since a prior reference point~\citep{BlocReed78, SahaSmit14}.  These findings suggest that the rapid contextual or situational changes that define these event boundaries shape how we experience and remember, similarly to how spatial boundaries~\citep[e.g.\ environmental barriers;][]{McKeBuzs16, BrunEtal18} shape how we experience and remember spatial environments and layouts, or how conceptual boundaries~\citep[e.g.\ distinctions between semantic categories;][]{BrunEtal18} shape how we experience and remember conceptual information.

Event boundaries are one example of the broader temporal covariance structure that is characteristic of ongoing experience.  Whereas classic approaches to studying memory encourage researchers to treat each moment as separable from the rest of experience, context-based theories of memory (including situation models and event-based models) posit that each moment derives meaning through its deep ties to other experiences.  This rich tapestry of interacting moments and experiences forms the scaffolding for interpreting our new experiences and for retrieving information concerning our past.  According to this view, the reactivation of this rich tapestry of contextual details enfolding the remembered event (i.e.\ how the event relates to the rest of ones experiences across timescales) is even more central to our subjective experience than the specific details of what occurred during that event!  This begs the question: what does it mean to \textit{remember} our past experiences?

\section*{The shape of remembered experience}

\begin{itemize}
\item Not about the number of details-- it's about capturing the overall ``shape'' of past experience
\item Quantifying recall should be about the match between the shape of the original experience (i.e. the covariance structure describing how it unfolds over time), not about the number of details that are recovered
\item When we retrieve memories, we cast ourselves along the full trajectory simultaneously (perhaps covering each component to a different degree), not to a single point (moment)
\item Predictions:
\begin{itemize}
    \item What makes an effective summary?  One that efficiently captures the overall shape of the experience.
    \item How do we communicate our experiences to others?  Describe the path through conceptual space.
    \item How do we communicate our experiences to our ``future selves?''  Describe the path through conceptual space.
    \item Each of these notions is fundamentally about the full trajectory, not about a single point.  Each moment derives meaning only in the context of the rest of experience.
\end{itemize}
\item What can list learning tell us?  We get fine control over  contextual representations at a very limited set of timescales.  We can study the detailed impacts of manipulations on memory.
\item What can't list learning tell us?  We can ask whether people precisely reproduce their prior experiences.  But we can't know whether they recover the ``gist'' of those experiences, since random lists do not have meaningful gists.
\item Other thoughts (from previous brainstorms):
    \begin{itemize}
    \item Reactivated neural representations that seem like noise should actually have predictive power for later memories if they get incorporated into the mental state at the time of remembering.
    \item List learning experiments and other experiments with discrete trials are not well equipped to study time spreading because they effectively impose event boundaries around and between every stimulus presentation.  Rather, we need continuous naturalistic recall experiments to more accurately reflect how our brains operate in the real world.
    \item say something about slow drift vs. reinstatement (cite MannEtal11 and FolkEtal18)
    \item something about schema learning (related to semantic memory?) and how memories are compressed (Chen et al 2017, Baldassano et al 2017, 2018) and transmitted to others
    \end{itemize}
\end{itemize}


Our brains exhibit autocorrelated activity.  In the extreme this is trivially true; our brains are biophysical systems that cannot change infinitely quickly, so neural activity patterns are necessarily similar (to some extent) from one moment to the next.  However, autocorrelated activity patterns \textit{at the time scales we typically examine in laboratory memory studies} are far stronger and persist for far longer than is strictly necessary from a biophysical standpoint.  For example, transient changes in neuronal membrane potentials during action potentials can last just a few milliseconds~\citep{HodgHuxl52} and even local field potentials that reflect the synchronized activities of tens of thousands of neurons can exhibit millisecond timescale changes~\citep{Frie05, Buzs06}.  Nevertheless, our brain activity pattern can remain reliably autocorrelated at timescales of seconds or longer-- and these long timescale autocorrelations predict how our ongoing experiences are remembered later~\citep[e.g.][]{MannEtal07, MannEtal11, HowaEtal12, FolkEtal18}.

\section*{Implications}
\subsection*{Extensions to spatial and semantic memory}
Just as we mentally spread ourselves across a distribution of times, we can perform similar mental operations when we conceptualize and retrieve other types of information.  For example, consider how we use pattern classifiers to connect neural activity patterns to specific semantic~\citep{semantic decoding studies} or spatial~\citep{spatial decoding studies} content being mentally manipulated.  The prevailing approach is to label specific brain activity patterns as each reflecting \textit{one particular} semantic concept or spatial location.  Variability in neural activity over repeated experiences or rememberings can result in classifiers spreading their predictions over multiple possible mental states.  Traditional approaches then ``clean up'' this distribution over possible mental states by taking its expected or maximum value.  However, following the ``quantum wave function'' logic, I suggest that the specific shape of these distributions may not only be informative, but they are \textit{more accurate reflections of what our neural activity patterns actually reflect}.

\section*{Concluding remarks}

\bibliographystyle{apa}
\bibliography{CDL-bibliography/memlab}
\end{document}
