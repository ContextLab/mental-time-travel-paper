\item Event boundaries: these shape the distribution because they sharply manipulate similarity as a function of time (whereas within an event similarity varies smoothly with time)
\item These ideas should affect time judgments: biases and variability in time judgments should reflect the shape of the memory distribution


At its core, characterizing episodic memory as mental time travel makes studying and understanding memory about matching up the thoughts (and/or brain activity patterns) participants are experiencing at one moment with the thoughts (and/or brain activity patterns) the experienced at another (e.g.\ prior) moment.

When experience is changing monotonically (i.e.\ in expectation, similarity in experience falls off monotonically with temporal distance, as during random word list experiments) this blurring out over time 

\item This view elegantly consolidated years of memory theory (cites) and has shaped much of the modern literature on episodic memory since
\item Context-based memory models (SAM, TCM, derivatives) say that when we retrieve past mental states we reactivate a mixture of prior moments, often centered around the time of the remembered episode.
\item Theoretical advantage: this spreading allows us to connect nearby moments in time
\item Alternative interpretation: what we actually experience when we recollect the past is  not a specific moment, but rather a weighted blend of moments.  This is how we model memory retrieval, so we should discuss memory retrieval as a distribution rather than as jumping back in time.