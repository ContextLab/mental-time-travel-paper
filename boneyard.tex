\textbf{KEY QUESTIONS}:
\begin{enumerate}
  \item Is this more of a ``review'' paper, or should I highlight the new analyses more?
  \item Who is this paper alienating and how can the narrative be reframed in a more inclusive way?
  \item Which ideas are ``new'' vs. restatements of old ideas?
  \item What are the key testable predictions to highlight, and how can they be communicated most clearly and effectively
  \item What's an appropriate venue for this?
\end{enumerate}



% Next to expand on:
% - Option 1: go into more depth about what the reinstatement functions look like for mental time travel vs. "quantum wave functions," under naturalistic vs. list-learning experiments.  Claim: different for naturalistic experiments but similar for list-learning experiments.
% - Option 2: get rid of the list learning stuff and transition directly to the "shape" of how experiences unfold.




\section*{How do our experiences and memories unfold over time?}
Although real-world experiences can exhibit rich temporal structures, it can be illustrative to consider the transformation of how we experience much simpler and less naturalistic events (e.g. studying random word lists) versus how we later remember those experiences.  Figure~\ref{fig:corrmats}A displays an average temporal correlation matrix describing the similarity structure of 1072 random word lists, each comprising 16 words~\citep{ZimaEtal18}.  I applied a topic model (trained on a curated corpus of Wikipedia articles) to yield a 100-dimensional embedding of each presented word~\citep{BleiEtal03}.  The left panel displays the average correlations between the topic vectors of each pair of presented words as a function of their presentation positions.  As shown, nearly all of the correlation ``mass'' is along the diagonal, and the off-diagonal entries are very close to zero.  Because the word orders were randomized, the temporal correlation matrix reflects that there are no consistent similarity patterns between any given pair of timepoints.  The right panel of Figure~\ref{fig:corrmats}A displays the average temporal correlation matrices describing participants' \textit{recalls} of those same lists.  Although the recall matrix exhibits the strongest correlations along its diagonal, some correlation mass bleeds into nearby recalls, reflecting (in part) participants' tendancies to recall semantically related words successively.\footnote{Note that participants recalled different numbers of words from each studied list; I used linear interpolation to resample each recall correlation matrix that preserved the size of the presentation correlation matrix.  This process resulted in some additional blurring along the diagonal.  In other words, the blurring along the diagonal reflects two phenomena: participants' tendancies to semanticly cluster their recalls, and an artifact of the temporal resampling.}  Differences in the temporal correlation structure between the original lists of words and the orders of participants' recalls illustrate one way that our memory systems can reorganize our experiences such that we recall them in a different way than we originally experienced them.







Further, this finding would appear to be at odds with classic context-based accounts of memory, to the extent that these models proport to explain memory associations between thematically similar experiences.  Specifically, if thematic overlap between temporally distant moments of the episode is rare, what 


\item Event boundaries: these shape the distribution because they sharply manipulate similarity as a function of time (whereas within an event similarity varies smoothly with time)
\item These ideas should affect time judgments: biases and variability in time judgments should reflect the shape of the memory distribution


At its core, characterizing episodic memory as mental time travel makes studying and understanding memory about matching up the thoughts (and/or brain activity patterns) participants are experiencing at one moment with the thoughts (and/or brain activity patterns) the experienced at another (e.g.\ prior) moment.

When experience is changing monotonically (i.e.\ in expectation, similarity in experience falls off monotonically with temporal distance, as during random word list experiments) this blurring out over time 

\item This view elegantly consolidated years of memory theory (cites) and has shaped much of the modern literature on episodic memory since
\item Context-based memory models (SAM, TCM, derivatives) say that when we retrieve past mental states we reactivate a mixture of prior moments, often centered around the time of the remembered episode.
\item Theoretical advantage: this spreading allows us to connect nearby moments in time
\item Alternative interpretation: what we actually experience when we recollect the past is  not a specific moment, but rather a weighted blend of moments.  This is how we model memory retrieval, so we should discuss memory retrieval as a distribution rather than as jumping back in time.