\title{mental time travel letter}
%
% See http://texblog.org/2013/11/11/latexs-alternative-letter-class-newlfm/
% and http://www.ctan.org/tex-archive/macros/latex/contrib/newlfm
% for more information.
%
\documentclass[11pt,stdletter,orderfromtodate,sigleft]{newlfm}
\usepackage{blindtext, xfrac, animate, hyperref, pxfonts, geometry}

  \setlength{\voffset}{0in}

\newlfmP{dateskipbefore=0pt}
\newlfmP{sigsize=20pt}
\newlfmP{sigskipbefore=10pt}
 
\newlfmP{Headlinewd=0pt,Footlinewd=0pt}
 
\namefrom{\vspace{-0.3in}Jeremy R. Manning}
\addrfrom{
	Dartmouth College\\
    %Department of Psychological \& Brain Sciences\\
    %HB 6207 Moore Hall\\
	Hanover, NH  03755}
 
\addrto{}
\dateset{\today}
 
\greetto{To the editors of \textit{Psychological Review}:}


 
\closeline{Sincerely,}
%\usepackage{setspace}
%\linespread{0.85}
% How will your work make others in the field think differently and move the field forward?
% How does your work relate to the current literature on the topic?
% Who do you consider to be the most relevant audience for this work?
% Have you made clear in the letter what the work has and has not achieved?

\begin{document}
\begin{newlfm}
I have enclosed my manuscript entitled \textit{Episodic memory: mental time travel or a quantum `memory wave' function?} to be considered for publication as an \textit{Article}.  The manuscript provides a new perspective on what it means to remember.

Tulving (1983) described episodic memory as a sort of ``mental time travel,'' whereby we send some part of our mental state back along our autobiographical timeline to a specific moment from our past.  This view (and other similar perspectives) have shaped how memory research is carried out in the laboratory, and how results are interpreted.  In my manuscript I argue that this framing misses fundamental aspects of where we ``go'' when we remember.  Rather than revisiting any particular moment from our past recent neural, behavioral, and theoretical evidence suggests that remembering entails casting ourselves back to a \textit{distribution} of moments from our past.  This provides one means of integrating information acquired during experiences that are not temporally contiguous.  In addition to re-framing episodic memory in the context of the recent literature, I outline the implications of these ideas with respect to studying the neural and behavioral basis of episodic memory.  I also use several analyses of previously published data to provide concrete examples that permeate the manuscript.

I expect that this article will be of interest to memory researchers and theorists, as well as to scientists interested in how we distribute thoughts over representational spaces when we carry out a variety of mental operations.

Thank you for considering this manuscript, and I hope you will find it suitable for publication in \textit{Psychological Reveiew}.


\end{newlfm}
\end{document}
