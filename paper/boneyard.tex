For example, the \textit{remember-know paradigm}~\citep{Tulv85b} asks participants to distinguish a set of probe items according to whether they \textit{remember} specific (episodic) details about the experience of having studied each item previously in the experiment, or whether they feel that they \textit{know} generally that they had studied the item but are unable to recollect specific details.  L


\textbf{KEY QUESTIONS}:
\begin{enumerate}
  \item Is this more of a ``review'' paper, or should I highlight the new analyses more?
  \item Who is this paper alienating and how can the narrative be reframed in a more inclusive way?
  \item Which ideas are ``new'' vs. restatements of old ideas?
  \item What are the key testable predictions to highlight, and how can they be communicated most clearly and effectively
  \item What's an appropriate venue for this?
\end{enumerate}


\begin{itemize}
\item Not about the number of details-- it's about capturing the overall ``shape'' of past experience
\item Quantifying recall should be about the match between the shape of the original experience (i.e., the covariance structure describing how it unfolds over time), not about the number of details that are recovered
\item When we retrieve memories, we cast ourselves along the full trajectory simultaneously (perhaps covering each component to a different degree), not to a single point (moment)
\item Predictions:
\begin{itemize}
    \item What makes an effective summary?  One that efficiently captures the overall shape of the experience.
    \item How do we communicate our experiences to others?  Describe the path through conceptual space.
    \item How do we communicate our experiences to our ``future selves?''  Describe the path through conceptual space.
    \item Each of these notions is fundamentally about the full trajectory, not about a single point.  Each moment derives meaning only in the context of the rest of experience.
\end{itemize}
\item What can list learning tell us?  We get fine control over  contextual representations at a very limited set of timescales.  We can study the detailed impacts of manipulations on memory.
\item What can't list learning tell us?  We can ask whether people precisely reproduce their prior experiences.  But we can't know whether they recover the ``gist'' of those experiences, since random lists do not have meaningful gists.
\item Other thoughts (from previous brainstorms):
    \begin{itemize}
    \item Reactivated neural representations that seem like noise should actually have predictive power for later memories if they get incorporated into the mental state at the time of remembering.
    \item List learning experiments and other experiments with discrete trials are not well equipped to study time spreading because they effectively impose event boundaries around and between every stimulus presentation.  Rather, we need continuous naturalistic recall experiments to more accurately reflect how our brains operate in the real world.
    \item say something about slow drift vs. reinstatement (cite MannEtal11 and FolkEtal18)
    \item something about schema learning (related to semantic memory?) and how memories are compressed (Chen et al 2017, Baldassano et al 2017, 2018) and transmitted to others
    \end{itemize}
  \end{itemize}


  \section*{BRAINSTORMING OUTLINE OF THE REST OF THE PAPER}
\begin{enumerate}
  \item \textit{Show video correlation matrix + recall correlation matrix.  Key idea: off-diagonal mass is introduced (primarily) by how people \textit{remember} the movie, not due to the contents of the movie itself (evidence: correlation matrices).  Further, this is not simply a matter of imprecision in how people recall, and/or our approach's ability to characterize recalled events (evidence: reinstatement figure showing which scenes are recalled similarly).  This sort of integration is how we ascribe meaning to events that unfold over disjointed extended timescales.}
  \item \textit{In theory this could also happen during \textit{any} sort of experience.  For example, when people study random word lists, they often re-order their recalls so that semantically related words are closer in the recall sequence than they were in the presented sequence.  This can be captured to an extent by semantic clustering measures.  However, it's difficult to examine this phenomenon in more depth because word lists necessarily lack the richness of more naturalistic stimuli.  There are no ``deeper ties'' between study events (i.e., individual words), so the sorts of integration operations that allow us to make sense of real-world experiences aren't given the same freedom of expression in list-learning experiments.  Evidence: presentation/recall correlation matrices during random free recall don't show the same sort of off-diagonal increase.  Claim: mental time travel and quantum wave reinstatement look the same for random word list remembering, but very different for real-world remembering.}
  \item \textit{The specific ties between unfolding events forms a scaffolding for understanding how those events are related.  This can be visualized geometrically by projecting word embedding representations of the events onto lower dimensional spaces (although see Flatland Fallacy paper, with HyperTools paper as a potential counterpoint).  This reveals another key property of how we remember: although there are substantial individual differences in the specific words people use to describe a shared experience, and how much detail they get into, the overall ``shapes'' of those recalled experience trajectories are highly similar across people.  This has potentially exciting implications for how we communicate memories to other people: we need to convey the ``gist'' (shape) of our experience.  It also has implications for how we learn by analogy: perhaps events that share schemas (cite Baldassano et al) unfold in similar ways (e.g. have similar shapes).  Learning which shapes go with which schemas could provide a scaffolding for rapid learning and generalization of new experiences, to the extent that those new experiences match a previously learned schema (shape).}
    \item These ideas also have implications for other forms of learning (e.g. spatial learning) and neural decoding.  When we attempt to characterize thoughts (e.g. training a classifier to distinguish between states), we tend to train decoders to estimate the ``one'' state that the brain is in.  However, perhaps we should instead consider that our brains are contemplating, or holding in our thoughts, many possible states simultaneously.  Therefore we should consider the full distribution of states (rather than only considering its maximum) as a primary metric of characterizing mental function.  (Probably people have proposed this-- e.g. representing the posterior and using it to compute...need citations.)
    \end{enumerate}

% Next to expand on:
% - Option 1: go into more depth about what the reinstatement functions look like for mental time travel vs. "quantum wave functions," under naturalistic vs. list-learning experiments.  Claim: different for naturalistic experiments but similar for list-learning experiments.
% - Option 2: get rid of the list learning stuff and transition directly to the "shape" of how experiences unfold.

    \textbf{JRM NOTE: add section on the temporal structure of list-learning vs. naturalistic stimulus, and on the temporal structure of recalls.}

% analysis idea: matrix of reinstatement functions for each timepoint-- e.g. considering each timepoint in turn (rows), how much are participants (on average) using language that mirrors how they talk about each other timepoint (columns).  there should be a strong diagonal, with some blockiness, and possibly some disjoint groups showing how events cluster.

    Our brains exhibit autocorrelated activity.  In the extreme this is trivially true; our brains are biophysical systems that cannot change infinitely quickly, so neural activity patterns are necessarily similar (to some extent) from one moment to the next.  However, autocorrelated activity patterns \textit{at the time scales we typically examine in laboratory memory studies} are far stronger and persist for far longer than is strictly necessary from a biophysical standpoint.  For example, transient changes in neuronal membrane potentials during action potentials can last just a few milliseconds~\citep{HodgHuxl52} and even local field potentials that reflect the synchronized activities of tens of thousands of neurons can exhibit millisecond timescale changes~\citep{Frie05, Buzs06}.  Nevertheless, our brain activity pattern can remain reliably autocorrelated at timescales of seconds or longer-- and these long timescale autocorrelations predict how our ongoing experiences are remembered later~\citep[e.g.][]{MannEtal07, MannEtal11, HowaEtal12, FolkEtal18}.




\section*{How do our experiences and memories unfold over time?}
Although real-world experiences can exhibit rich temporal structures, it can be illustrative to consider the transformation of how we experience much simpler and less naturalistic events (e.g. studying random word lists) versus how we later remember those experiences.  Figure~\ref{fig:corrmats}A displays an average temporal correlation matrix describing the similarity structure of 1072 random word lists, each comprising 16 words~\citep{ZimaEtal18}.  I applied a topic model (trained on a curated corpus of Wikipedia articles) to yield a 100-dimensional embedding of each presented word~\citep{BleiEtal03}.  The left panel displays the average correlations between the topic vectors of each pair of presented words as a function of their presentation positions.  As shown, nearly all of the correlation ``mass'' is along the diagonal, and the off-diagonal entries are very close to zero.  Because the word orders were randomized, the temporal correlation matrix reflects that there are no consistent similarity patterns between any given pair of timepoints.  The right panel of Figure~\ref{fig:corrmats}A displays the average temporal correlation matrices describing participants' \textit{recalls} of those same lists.  Although the recall matrix exhibits the strongest correlations along its diagonal, some correlation mass bleeds into nearby recalls, reflecting (in part) participants' tendancies to recall semantically related words successively.\footnote{Note that participants recalled different numbers of words from each studied list; I used linear interpolation to resample each recall correlation matrix that preserved the size of the presentation correlation matrix.  This process resulted in some additional blurring along the diagonal.  In other words, the blurring along the diagonal reflects two phenomena: participants' tendancies to semanticly cluster their recalls, and an artifact of the temporal resampling.}  Differences in the temporal correlation structure between the original lists of words and the orders of participants' recalls illustrate one way that our memory systems can reorganize our experiences such that we recall them in a different way than we originally experienced them.







Further, this finding would appear to be at odds with classic context-based accounts of memory, to the extent that these models proport to explain memory associations between thematically similar experiences.  Specifically, if thematic overlap between temporally distant moments of the episode is rare, what 


\item Event boundaries: these shape the distribution because they sharply manipulate similarity as a function of time (whereas within an event similarity varies smoothly with time)
\item These ideas should affect time judgments: biases and variability in time judgments should reflect the shape of the memory distribution


At its core, characterizing episodic memory as mental time travel makes studying and understanding memory about matching up the thoughts (and/or brain activity patterns) participants are experiencing at one moment with the thoughts (and/or brain activity patterns) the experienced at another (e.g.\ prior) moment.

When experience is changing monotonically (i.e.\ in expectation, similarity in experience falls off monotonically with temporal distance, as during random word list experiments) this blurring out over time 

\item This view elegantly consolidated years of memory theory (cites) and has shaped much of the modern literature on episodic memory since
\item Context-based memory models (SAM, TCM, derivatives) say that when we retrieve past mental states we reactivate a mixture of prior moments, often centered around the time of the remembered episode.
\item Theoretical advantage: this spreading allows us to connect nearby moments in time
\item Alternative interpretation: what we actually experience when we recollect the past is  not a specific moment, but rather a weighted blend of moments.  This is how we model memory retrieval, so we should discuss memory retrieval as a distribution rather than as jumping back in time.