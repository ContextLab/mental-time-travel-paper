\documentclass[12pt]{article}
\usepackage[utf8]{inputenc}
\usepackage{geometry}
\usepackage{natbib}
\usepackage{pxfonts}
\usepackage{graphicx}

\title{Episodic memory: mental time travel or a quantum `memory wave' function?}
\author{Jeremy R. Manning\\Dartmouth College, Hanover, NH\\jeremy.r.manning@dartmouth.edu}

\begin{document}
\section*{Proposed manuscript information}
\begin{itemize}
  \item \textbf{Title:} Episodic memory: mental time travel or a quantum `memory wave' function?
  \item \textbf{Author:} Jeremy R. Manning
  \item \textbf{Affiliation:} Department of Psychological and Brain Sciences, Dartmouth College
  \item \textbf{Email address:} jeremy.r.manning@dartmouth.edu
  \item \textbf{Phone number:} 603-646-2777
\end{itemize}

\section*{Proposed summary}
Where do we ``go'' when we recollect our past?  When remembering a past event, it is intuitive to imagine some part of ourselves mentally ``jumping back in time'' to when the event occurred. I propose an alternative view, inspired by recent evidence from my lab and others, as well as by re-examining existing models of episodic recall, that suggests that this notion of mentally revisiting any specific moment of our past is at best incomplete and at worst misleading.  Instead, I suggest that we retrieve information from our past by mentally casting ourselves back simultaneously to \textit{many} time points from our past, much like a quantum wave function spreading its probability mass over many possible states.  This revised conceptual model makes important behavioral and neural predictions about how we retrieve information about our past, and has implications for how we study episodic memory experimentally.

\section*{Timeliness of this work}
Over the past several years, the episodic memory literature has grown along several key fronts, ranging from better characterizations of the neural basis of the formation and retrieval of episodic memories, to advances in characterizations of episodic memory behaviors.  Alongside these advances has been a renewed focus on experiments that incorporate the use of complex ``naturalistic'' stimuli such as videos, natural images, and recorded speech.  Further, several recent studies have eschewed traditional trial-based designs in favor of less tightly controlled ``free form'' experiments.  Finally, advances in machine learning (especially deep learning) and related neuropsychological fields (especially spatial navigation) have lent some additional perspective to how information might be represented from a functional standpoint, and retreived during memory search.  Taken together, I believe it is time to reconsider how we characterize and interpret a number of old findings in the ``classic'' episodic memory literature.  Specifically, I propose that Tulving's notion of \textit{mental time travel} that has driven progress over the past several decades is ready to be refreshed in light of these new findings and approaches.

\nocite{AlyEtal18, BaldEtal17, BaldEtal18, BrunEtal18, ChenEtal17, FolkEtal18, LohnEtal18, MauEtal18, McKeBuzs16, Rang18, TsaoEtal18, BrigEtal18, GatyEtal16, IsolEtal17, JollChan18, LongKaha18, StYvNase18}

\renewcommand\refname{Key recent papers}
\bibliographystyle{apa}
\bibliography{CDL-bibliography/memlab}
\end{document}
